%% This is the main file and you use this file to organize your assignment.

\documentclass[a4paper]{article}	  
\usepackage[margin=3cm]{geometry} 	   % Choose your margin here. 
\usepackage{amsmath}
\usepackage{parskip}
\usepackage{graphicx}
\usepackage{caption}
\usepackage{subcaption}

\newcommand{\figref}[1]{\figurename~\ref{#1}}

\let\endtitlepage\relax						% Begin the text immidiately after the title page. Optional
\setlength{\parindent}{0cm}				% Start paragraph without indent. Optional

\begin{document}

\begin{titlepage}
\begin{center}
\Large TTK4190 Guidance and Control of Vehicles \\
\vspace{10pt}
\Large \LaTeX{} Template Assignment 1 \\
\vspace{10pt}
\large Written Fall 2019 By Name
\end{center}
\end{titlepage}

\section*{Info}
This is only a short template for the assignments and it is definitely not necessary to use this template if you are familiar with \LaTeX{} beforehand. The best way to learn \LaTeX{} is to write some stuff yourself and Google problems you run into. Therefore, we will not answer \LaTeX{} related questions outside of the assignment guidance, but you are of course welcome to ask questions then.

\section{Attitude Control of Satellite}
\subsection{}
\subsubsection{Finding equilibrium}
We are given the following equations of motion (EoM) for the satellite
\begin{equation}
\label{eq:dynamics}
	\begin{aligned}
		\dot{\mathbf{q}} = \mathbf{T}_q (\mathbf{q} ) \boldsymbol{\omega}, \\
		\mathbf{I}_{CG} \dot{\boldsymbol{\omega}} - \mathbf{S} (\mathbf{I}_{CG} \boldsymbol{\omega} ) \boldsymbol{\omega} & =  \boldsymbol{\tau}.
	\end{aligned}
\end{equation}
To linearize this, we must first find an equilibrium point. We set the derivatives $\dot{\mathbf{q}}$ and $\dot{\boldsymbol{\omega}}$ equal to zero and in addition require $\mathbf{q} = [\eta, \epsilon_1, \epsilon_2, \epsilon_3]^\top  = [\eta, 0, 0, 0]^\top$ and $\boldsymbol{\tau} = \mathbf{0}$ thus obtaining
\begin{align}
\label{eq:ang_vel_trans_1}
\mathbf{T_q}(\mathbf{q})\boldsymbol{\omega}_0 = \mathbf{0}, \\
\label{eq:ang_vel_trans_2}
\mathbf{S}(\mathbf{I}_{CG} \boldsymbol{\omega}_0)\boldsymbol{\omega}_0 = \mathbf{0}.
\end{align}
Using (2.69) from \cite{Fossen2011} we can rewrite \eqref{eq:ang_vel_trans_1} and obtain
\begin{equation}\begin{aligned}
\mathbf{T_q}(\mathbf{q})\boldsymbol{\omega}_0
&= \frac{1}{2}
\begin{bmatrix}
- \boldsymbol{\epsilon}_0^\top \\
\eta \mathbf{I}_3 + \mathbf{S}(\boldsymbol{\epsilon}_0)
\end{bmatrix}
\boldsymbol{\omega}_0 \\
&= \frac{1}{2}
\begin{bmatrix}
\mathbf{0}^\top \\
\mathbf{I}_3
\end{bmatrix}
\boldsymbol{\omega}_0 \\
&= \mathbf{0}.
\end{aligned}\end{equation}
Hence $\boldsymbol{\omega}_0 = \mathbf{0}^\top$ and we have the equilibrium point $\mathbf{x}_0 = [\boldsymbol{\epsilon}_0^\top, \boldsymbol{\omega}_0^\top]^\top = \mathbf{0}^\top$. This clearly also satisfies \eqref{eq:ang_vel_trans_2}.
\subsubsection{Linearizing around equilibrium}
We wish to linearize this non-linear system around $\mathbf{x}_0$. For that, we need expressions for $\dot{\boldsymbol{\epsilon}}$ and $\dot{\boldsymbol{\omega}}$. We find
\begin{align}
\dot{\boldsymbol{\epsilon}}
&= \frac{1}{2}(\eta \mathbf{I}_3 + \mathbf{S}(\boldsymbol{\epsilon}))\boldsymbol{\omega}
= \frac{1}{2}\eta \boldsymbol{\omega} + \frac{1}{2}\mathbf{S}(\boldsymbol{\omega})\boldsymbol{\omega}
= \frac{1}{2}\eta \boldsymbol{\omega}
= \frac{1}{2}\sqrt{1 - \boldsymbol{\epsilon}^\top\boldsymbol{\epsilon}}\boldsymbol{\omega}, \quad \text{and} \\
\dot{\boldsymbol{\omega}}
&= \mathbf{I}_{CG}^{-1}(\mathbf{S}(\mathbf{I}_{CG}\boldsymbol{\omega})\boldsymbol{\omega} + \boldsymbol{\tau})
=\frac{1}{mr^2}(\mathbf{S}(\mathbf{I}_{CG}\boldsymbol{\omega})\boldsymbol{\omega} + \boldsymbol{\tau})
= \boldsymbol{S}(\boldsymbol{\omega})\boldsymbol{\omega} + \frac{1}{mr^2}\boldsymbol{\tau}
= \frac{1}{mr^2}\boldsymbol{\tau}.
\end{align}
So,
\begin{equation}\begin{aligned}
\dot{\mathbf{x}} =
\begin{bmatrix}
\dot{\boldsymbol{\epsilon}}\\
\dot{\boldsymbol{\omega}}\\
\end{bmatrix}
=\begin{bmatrix}
\frac{1}{2}\sqrt{1-\boldsymbol{\epsilon}^\top\boldsymbol{\epsilon}}\boldsymbol{\omega}\\
\frac{1}{mr^2}\boldsymbol{\tau}\\
\end{bmatrix}
= \mathbf{f}(\mathbf{x}, \boldsymbol{\tau}).
\end{aligned}\end{equation}
The linearized system has the form
\begin{equation}\begin{aligned}
\dot{\hat{\mathbf{x}}} = \mathbf{A}\hat{\mathbf{x}} + \mathbf{B}\hat{\boldsymbol{\tau}}
\end{aligned}\end{equation}
where $\mathbf{A} = $ and $\mathbf{B}$ are the Jacobians $\frac{\partial \mathbf{f}}{\partial \mathbf{x}}$, and $\frac{\partial \mathbf{f}}{\partial \boldsymbol{\tau}}$ respectively, evaluated at the equilibrium $\mathbf{x}=\mathbf{x_0}$. That is,
\begin{align}
\label{eq:A_matrix}
\mathbf{A}
&= \begin{bmatrix}
-\frac{1}{2}\sqrt{1-\boldsymbol{\epsilon}_0^\top\boldsymbol{\epsilon}_0}\boldsymbol{\epsilon}_0\boldsymbol{\omega}_0^\top & \frac{1}{2}\sqrt{1-\boldsymbol{\epsilon}_0^\top\boldsymbol{\epsilon}_0}\mathbf{I}_3\\
\mathbf{0}_3 & \mathbf{0}_3 \\
\end{bmatrix}
= \begin{bmatrix}
\mathbf{0}_3 & \frac{1}{2}\mathbf{I}_3 \\
\mathbf{0}_3 & \mathbf{0}_3 \\
\end{bmatrix}, \quad \text{and} \\
\mathbf{B}
\label{eq:B_matrix}
&= \begin{bmatrix}
\mathbf{0}_3\\
\frac{1}{mr^2}\mathbf{I}_3\\
\end{bmatrix}
\end{align}
where $\mathbf{0}_3$ denotes a 3-by-3 zero-valued matrix. \eqref{eq:A_matrix} follows from the following results
\begin{align}
\frac{\partial}{\partial \epsilon_i}\frac{1}{2}\sqrt{1-\epsilon_1^2 - \epsilon_2^2 - \epsilon_3^2}\omega_k
&= -\frac{1}{2}\sqrt{1-\epsilon_1^2 - \epsilon_2^2 - \epsilon_3^2}^{-1}\epsilon_i \omega_k, \\
\frac{\partial}{\partial \omega_i}\frac{1}{2}\sqrt{1-\boldsymbol{\epsilon}^\top\boldsymbol{\epsilon}}\omega_i
&=\frac{1}{2}\sqrt{1-\boldsymbol{\epsilon}^\top\boldsymbol{\epsilon}}.
\end{align}


\subsection{}
We now introduce the control law
\begin{equation}
  \label{eq:tau}
  \mathbf{\tau} = -\mathbf{K}_d \boldsymbol{\omega} - k_p \boldsymbol{\epsilon},
\end{equation}
where $\mathbf{K}_d = k_d \mathbf{I}_3 = 40 \mathbf{I}_3$ and $k_p = 2$. Since this input is only dependent on our state vector $x$, we can find an expression for the closed loop system with a single system matrix. This system becomes
\begin{equation}\begin{aligned}
\dot{\hat{ \mathbf{x}}}
&= \mathbf{A}\hat{\mathbf{x}} + \mathbf{B}\hat{\boldsymbol{\tau}} \\
&= \mathbf{A}\hat{\mathbf{x}} + \mathbf{B}(-\mathbf{K}_d \boldsymbol{\omega} - k_p \boldsymbol{\epsilon}) \\
&= \mathbf{A}\hat{\mathbf{x}} + \mathbf{B}(
-\begin{bmatrix}
\mathbf{0}_3 & \mathbf{K}_d \\
\end{bmatrix} \hat{\mathbf{x}}
-\begin{bmatrix}
k_p \mathbf{I}_3 & \mathbf{0}_3 \\
\end{bmatrix}\hat{\mathbf{x}}
) \\
&= (\mathbf{A}\hat{\mathbf{x}} - \mathbf{B}
\begin{bmatrix}
k_p \mathbf{I}_3 & \mathbf{K}_d \\
\end{bmatrix}
) \hat{\mathbf{x}} \\
&= \left(
\begin{bmatrix}
\mathbf{0}_3 & \frac{1}{2}\mathbf{I}_3 \\
\mathbf{0}_3 & \mathbf{0}_3 \\
\end{bmatrix}
-
\begin{bmatrix}
\mathbf{0}_3\\
\frac{1}{mr^2}\mathbf{I}_3\\
\end{bmatrix}
\begin{bmatrix}
k_p\mathbf{I}_3 & \mathbf{K}_d \\
\end{bmatrix}
\right) \hat{\mathbf{x}} \\
&=
\begin{bmatrix}
\mathbf{0}_3 & \frac{1}{2}\mathbf{I}_3 \\
-\frac{k_p}{mr^2}\mathbf{I}_3& -\frac{1}{mr^2}\mathbf{K}_d \\
\end{bmatrix}
\hat{\mathbf{x}}.
\end{aligned}\end{equation}
The matlab command below is used to find the eigenvalues of this matrix, which will coincide with the poles of the closed loop transfer function.
\lstset{language=Matlab, basicstyle=\small}
\begin{lstlisting}[frame=single]
eigs([zeros(3,3), 1/2*eye(3); -k_p/(m*r^2)*eye(3), -k_d/(m*r^2)*eye(3)])
\end{lstlisting}
These eigenvalues are found to be $\lambda_{1,2} = -0.0278 \pm 0.0248j$, i.e. complex conjugated in the left half plane. This means that the system is stable. \textbf{TODO: Do we want complex conjugated or real?}

\subsection{}
Equation (2) from the assignment can be written as:
\begin{equation}
  \label{eq:tau}
  \mathbf{\tau} = -\mathbf{K}_d \boldsymbol{\omega} - k_p \boldsymbol{\epsilon}
\end{equation}

\subsection{}
The quaternion error can be written as
 \begin{equation}
	 \tilde{\mathbf{q}} := \left[
	 \begin{array}{c}
		 \tilde{\eta} \\
		 \tilde{\epsilon}
	 \end{array}
	 \right] = \bar{\mathbf{q}}_d \otimes \mathbf{q}
 \end{equation}

\subsection{Problem 1.5}
In problems with simulations, you need to include figures in the report:
\begin{figure}[ht]
	\centering
	\includegraphics[width=0.7\textwidth]{fig1} % Filename is "fig1.png" and must be located in the same folder as this file. If you have a folder containing all the figures you can use "Figures/fig 1" as long as the "Figures" folder is placed in the same folder as this file.
	\caption{Figure of something useful.}
	\label{fig:fig1}
\end{figure}

You can now refer to this figure as \figref{fig:fig1}. You can also insert figures side-by-side as in Figure \ref{fig:2}. %Notice that \figref includes the word Figure before the reference. If you use "\ref", you need to write the word Figure yourself.
\begin{figure}[ht]
	\centering
	\begin{subfigure}[b]{0.45\textwidth}
		\includegraphics[width=\textwidth]{fig1}
		\caption{caption..}
		\label{fig:2a}
	\end{subfigure}
	~ %add desired spacing between images, e. g. ~, \quad, \qquad, \hfill etc.
	%(or a blank line to force the subfigure onto a new line)
	\begin{subfigure}[b]{0.45\textwidth}
		\includegraphics[width=\textwidth]{fig1}
		\caption{caption..}
		\label{fig:2b}
	\end{subfigure}
	\begin{subfigure}[b]{0.45\textwidth}
		\includegraphics[width=\textwidth]{fig1}
		\caption{caption..}
		\label{fig:2c}
	\end{subfigure}
	\begin{subfigure}[b]{0.45\textwidth}
		\includegraphics[width=\textwidth]{fig1}
		\caption{caption..}
		\label{fig:2d}
	\end{subfigure}
	\caption{Caption for all figures}\label{fig:2}
\end{figure}


\subsection{Problem 1.6}
The control law in this problem can be written as
\begin{equation}
	\boldsymbol{\tau} = -\mathbf{K}_d \tilde{\boldsymbol{\omega}} - k_p \tilde{\boldsymbol{\epsilon}}
\end{equation}
and the desired angular velocity as
\begin{equation}
	\boldsymbol{\omega}_d = \mathbf{T}^{-1}_{\Theta_d}(\Theta_d)\dot{\Theta}_d
\end{equation}

\subsection{Problem 1.7}
The Lyapunov function can be written as
 \begin{equation}
	 V = \frac{1}{2} \tilde{\boldsymbol{\omega}}^{\top} \mathbf{I}_{CG}\tilde{\boldsymbol{\omega}} + 2 k_p (1-\tilde{\eta})
 \end{equation}
and the derivative as
\begin{equation}
	\dot{V} = -k_d \boldsymbol{\omega}^{\top} \boldsymbol{\omega}
\end{equation}

% Note that \mathbf can be used for bold letters in math mode (within equations and dollar signs). \boldsymbol can be used to get bold greek letters.
	% Use "\include" instead of "\input" if you want the section to start on a new page. "problem1" is a tex file included at this location in the document. It is possible to answer the whole assignment in the main file (paste everything from "problem1.tex" and "problem2.tex" here), but that restricts the readability. Therefore, one file is created for each problem.
\section{Straight-line path following in the horizontal plane}
\subsection{}
From (2.25) we have: 
\begin{equation}
	\dot{\mathbf{p}}^b_{nb} = \mathbf{R}(\mathbf{\Theta_{nb}}) \mathbf{v}^b_{nb}
\end{equation}

Then: 
\begin{align*}
	\dot{x} &= u \cos(\psi) \cos (\theta) + v [\cos (\psi) \sin( \theta) \sin (\phi) - \sin (\psi) \cos (\theta)] \\
	&+ w [\sin (\psi) \sin (\phi) + \cos (\psi) \cos (\phi) \sin(\theta)] \\
	&= u \cos(\psi) * 1 + v [\cos (\psi) * 0 * 0 - \sin (\psi) * 1] + w [\sin (\psi) * 0 + \cos (\psi) * 1 * 0] \\
	&= u \cos(\psi) - v \sin(\psi) \\
	\dot{y} &= u \sin (\psi) \cos (\theta) 
	+ v [\cos (\psi) \cos (\phi) + \sin (\phi) \sin (\theta) \sin (\psi)] \\
	&+ w [\sin (\theta) \sin (\psi) \cos (\phi) + \cos(\psi) \sin(\phi)] \\ 
	&= u \sin (\psi) * 1
	+ v [\cos (\psi) * 1 + \sin (\phi) * 0 * 0] 
	+ w [0 * \sin (\psi) * 1 + \cos(\psi) * 0] \\ 
	&= u \sin (\psi) + v \cos (\psi) 
\end{align*}
Using: 
\begin{align}
	\sin (\arctan (x)) &= \frac{x}{1 + x^2} \\
	\cos (\arctan (x)) &= \frac{1}{1 + x^2} \\ 
	U &= \sqrt{u^2 + v^2} \\ 
	\beta &= \arctan(\frac{v}{u}) \\
	\chi &= \psi + \beta 
\end{align}

We have: 
\begin{align*}
	\dot{x} &= u \cos(\psi) - v \sin(\psi) 
	= \sqrt{u^2 + v^2} [\cos (\psi) \frac{u}{\sqrt{u^2 + v^2}} 
	- \sin (\psi) \frac{v}{\sqrt{u^2 + v^2}}] \\
	&= U [\cos (\psi) \frac{1}{\sqrt{1 + \frac{v^2}{u^2}}} 
	- \sin (\psi) \frac{\frac{v}{u}}{\sqrt{1 + \frac{v^2}{u^2}}}] \\ 
	&= U [\cos (\psi) \cos(\arctan(\frac{v}{u})) 
	- \sin (\psi) \sin(\arctan(\frac{v}{u}))] 
	= U [\cos (\psi) \cos(\beta)) - \sin (\psi) \sin(\beta))] \\ 
	&= U \cos(\psi + \beta) = U \cos(\chi) \\
	\dot{y} &= u \sin (\psi) + v \cos (\psi) 
	= \sqrt{u^2 + v^2} [\sin (\psi) \frac{u}{\sqrt{u^2 + v^2}}
	+ \cos (\psi) \frac{v}{\sqrt{u^2 + v^2}}] \\
	&= U [\sin (\psi) \frac{1}{\sqrt{1 + \frac{v^2}{u^2}}} 
	+ \cos (\psi) \frac{\frac{v}{u}}{\sqrt{1 + \frac{v^2}{u^2}}}] 
	= U [\sin (\psi) \cos (\arctan(\frac{v}{u}))
	+ \cos (\psi) \sin (\arctan(\frac{v}{u}))] \\
	&= U \sin(\psi + \arctan (\frac{v}{u})) 
	= U \sin(\psi + \beta) 
	= U \sin(\chi)
\end{align*}

\subsection{}
\begin{align}
	\dot{x} &= U \cos (\psi + \beta) \\
	\dot{y} &= U \sin (\psi + \beta) 
\end{align}
With $\beta$ small (and a disturbance) and $\psi$ also small. When $\psi$ is small, $\cos(\psi) \approx 0$ and $\sin(\psi) \approx \psi$. This can be seen from Taylor expansion of the sine and cosine functions for small values. 

As the vessel is following a straight line in the horizontal plane, and the assumtions above hold, it is clear that the only part of the cross-track error, $e(t) = -[x(t) - x_k]\sin(\alpha_k) + [y(t) - y_k]\cos(\alpha_k)$, which will affect it is $y$, such that we can give $e(t) = y$. 

\subsection{}
Transfer functions can be written as
\begin{equation}
	H(s) = \frac{a_n s^n + ... + a_1 s + a_0}{b_m s^m + ... + b_1 s + b_0}
\end{equation}

The Nomoto model can be written as
\begin{equation}
\label{eq:nomoto}
	\begin{aligned}
		T \dot{r} + r &= K \delta + b \\
		\dot{\psi} &= r
	\end{aligned}
\end{equation}

\subsection{}
Answer Problem 2.4 here.  References can be placed in the "bibliography.bib" and referred to as \cite{Fossen2011} and \cite{Fjellstad1994857}. The PID-controller is
\begin{equation}
	\delta = -k_p y - k_d \dot{y} - k_i \int y
\end{equation}

 
\bibliographystyle{IEEEtran}
\bibliography{bibliography.bib}

\end{document}